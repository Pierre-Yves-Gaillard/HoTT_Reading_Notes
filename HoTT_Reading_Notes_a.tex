% HoTT_Reading_Notes_a
% http://goo.gl/Dz6r8g
% https://docs.google.com/document/d/15-yaLYocktW8PtUeiKjTupslbpdnDs0HH5GIZA20Sxc/edit
% https://github.com/Pierre-Yves-Gaillard/HoTT_Reading_Notes 
% http://iecl.univ-lorraine.fr/~Pierre-Yves.Gaillard/HoTT/ReadingNotes
% https://docs.google.com/document/d/15-yaLYocktW8PtUeiKjTupslbpdnDs0HH5GIZA20Sxc/edit?usp=sharing
% !TEX encoding = UTF-8 Unicode
\documentclass[12pt]{article}
\usepackage[T1]{fontenc}
\usepackage[utf8]{inputenc}
\usepackage{amssymb,amsmath}
\usepackage[a4paper]{geometry}
\usepackage{datetime}
\addtolength{\parskip}{.5\baselineskip}
%\renewcommand{\baselinestretch}{1.1}
%\pagestyle{empty}
\newcommand{\cc}{\cdot}
\newcommand{\dd}{\displaystyle}
\newcommand{\ee}{\equiv}
%\newcommand{\la}{\lambda}
\newcommand{\msf}{\mathsf}
\newcommand{\nn}{\noindent}
\newcommand{\oo}{\operatorname}
\newcommand{\p}{\varphi}
\newcommand{\sm}{\scriptstyle}
\newcommand{\ap}{\mathsf{ap}}
\newcommand{\id}{\mathsf{id}}
\newcommand{\ind}{\mathsf{ind}}
\newcommand{\inl}{\mathsf{inl}}
\newcommand{\inr}{\mathsf{inr}}
\newcommand{\pr}{\mathsf{pr}}
\newcommand{\refl}{\mathsf{refl}}
\newcommand{\uniq}{\mathsf{uniq}}
\newcommand{\U}{\mathcal U}
\title{HoTT Reading Notes}
\author{Pierre-Yves Gaillard}
\date{\today, \currenttime}
\begin{document}
% $$\text{mathbf }\mathbf1,\text{ pmb }\pmb1,\text{ boldsymbol }\boldsymbol1$$ \tiny, \scriptsize, \footnotesize, \small, \normalsize, \large, \Large, \LARGE, \huge, \Huge

\maketitle%\nn\today, \currenttime

\nn This text is available at 

\nn{\small\verb"http://iecl.univ-lorraine.fr/~Pierre-Yves.Gaillard/HoTT/ReadingNotes"}

\nn\verb"https://github.com/Pierre-Yves-Gaillard/HoTT_Reading_Notes"

\nn\verb"http://goo.gl/Dz6r8g"

This is an informal set of comments on the HoTT Book:

\nn\verb"http://homotopytypetheory.org/book/"

This is work in progress.

\tableofcontents%\newpage

%%

\section{Exercise 1.3.} Derive the induction principle for products $\ind_{A\times B}$, using only the projections and the propositional uniqueness principle $\uniq_{A\times B}$. Verify that the definitional equalities are valid. Generalize $\uniq_{A\times B}$ to $\Sigma$-types, and do the same for $\Sigma$-types. \emph{(This requires concepts from Chapter~2.)}

\nn\emph{Solution.} It suffices to handle $\Sigma$-types. 

\nn(a) We assume that, for all type family $B:A\to\U$ we have a map 
$$
\pr_1:\left(\sum_{a:A}B(a)\right)\to A,
$$ 
a dependent function 
$$
\pr_2:\prod_{x:\sum_{a:A}B(a)}B(\pr_1(x))
$$
satisfying
\begin{equation}\label{131}
\pr_1(a,b)\equiv a,\quad\pr_2(a,b)\equiv b
\end{equation}
for all $a:A$ and all $b:B(a)$.

\nn(b) We also assume that, for all type family $B:A\to\U$ we have a dependent function 
$$
\uniq_{\sum_{a:A}B(a)}:\prod_{x:\sum_{a:A}B(a)}(\pr_1(x),\pr_2(x))=x
$$ 
such that 
\begin{equation}\label{132}
\uniq_{\sum_{a:A}B(a)}(a,b)\equiv\refl_{(a,b)}
\end{equation}
for all $a:A$ and all $b:B(a)$. 

\nn(c) We define
$$
\ind_{\sum_{a:A}B(a)} : \prod_{C:\sum_{a:A}B(a)\to\U} 
\left(\prod_{a:A}\prod_{b:B(a)} C(a,b)\right) 
\to \prod_{x:\sum_{a:A}B(a)} C(x)
$$
thanks to the transport principle, which is indeed a concept from Chapter~2, by
\begin{equation}\label{133}
\ind_{\sum_{a:A}B(a)}(C,g,x):
\equiv\uniq_{\sum_{a:A}B(a)}(x)_*\Big(g(\pr_1(x))(\pr_2(x))\Big).
\end{equation}
(d) We must prove 
$$
\ind_{\sum_{a:A}B(a)}(C,g,(a,b))\equiv g(a)(b)
$$
for all $a:A$ and all $b:B(a)$. But this follows from \eqref{131}, \eqref{132} and \eqref{133}.

%%

\section{Comment after Corollary 2.7.3}

If $w,w':\sum_{x:A}P(x)$, then 
$$
\msf{pair}^=:\left(\sum_{p:p_1w=p_1w'}(p_*p_2w=p_2w')\right)\to(w=w').
$$ 
(It would be more rigorous to write $\msf{pair}^=(w,w')$ instead of $\msf{pair}^=$.)

%%

\section{Proof of Theorem 2.11.1}

Let $f:A\to B$ be an equivalence, and let $a,a':A$. We must show that $\ap_f:(a=a')\to (fa=fa')$ is an equivalence.

Step 1. (This step is the the same as in the book; I spell it out for the reader's convenience.) Let $g:B\to A$ be a quasi-inverse to $f$, let $\beta:g\circ f=\id_A$, and define 
$$
\varphi:(gfa=gfa')\to(a=a')\quad\text{by}\quad\varphi(q):\equiv\beta_a^{-1}\cdot q\cdot\beta_{a'}.
$$ 
Note, as in the book, that the equality $\varphi\circ\ap_g\circ\ap_f=\id_{a=a'}$ follows from the functoriality of $\ap$ and the naturality of homotopies, Lemmas 2.2.2 and 2.4.3. (Here $\ap_g:(fa=fa')\to (gfa=gfa')$.)

Step 2. Since $\varphi$ is easily seen to be an equivalence, $\ap_g\circ\ap_f$ is also an equivalence, and so is, by symmetry, $\ap_f\circ\ap_g$. This implies that $\ap_f$ and $\ap_g$ are equivalences.

%old: https://docs.google.com/document/d/1OvMAEN0uvE18rTCHAmZAIYKiILQ_FNT5JJqNH3HH760/edit

%%

\section{Proof of Lemma 4.1.2}

The claim

\nn ``It remains to show that $h$ is identified with $h'$ when transported along this equality, which by transport in identity types and function types, reduces to showing 
$$
h(s)=h(p)\cdot h'(p)^{-1}\cdot h'(s)
$$ 
for any $s:a=x$''

\nn in the proof of Lemma 4.1.2 in the HoTT Book follows immediately from Theorem 2.7.2 and the Lemma below, which is easily proved by path induction:

\nn \textbf{Lemma.} Let $A$ be a type; let $a,x:A$; let $q:a=a$; let $P:(x=x)\to\U$ be defined by 
$$
P(r):\equiv\prod_{s:a=x}(r=s^{-1}\cdot q\cdot s);
$$ 
let $r,r':x=x,\ t:r=r',\ h:P(r),\ s:a=x$. Then 
$$
\oo{transport}^P(t,h)(s)=t^{-1}\cdot h(s).
$$ 

%%

\section{Proof of Lemma 4.1.2}

End of the proof: To complete the proof, we set $f(x):\ee\msf{pr}_1(k(x))$ with $k:\prod_{(x:A)}B(x)$.
%https://github.com/HoTT/book/blob/master/equivalences.tex#L107

%%

%\nn$\bullet$ Proof of Theorem 4.1.3.

%The fact that $\mathbf2\simeq\mathbf2$ is a set follows from univalence and Examples 3.1.2 and 3.1.5.

%%

\section{Proof of Theorem 4.7.6}

Let us check Equivalence $(*)$ in the proof of Theorem 4.7.6: For $a:A$ and $p:a=x$ put 
$$
Q(a,p):\equiv\sum_{u:P(a)}p_*(f(a,u))=v.
$$ 
We claim 
$$
\sum_{a:A}\sum_{p:a=x}Q(a,p)\overset{(1)}{\simeq}
\sum_{q:\sum_{a:A}(a=x)}Q(q)\overset{(2)}{\simeq}
Q(x,\oo{refl}_x).
$$
Equivalence (1) follows from Exercise 2.10. As Lemma 3.11.8 implies that $$\sum_{a:A}\ (a=x)$$ is contractible with center $(x,\oo{refl}_x)$, Equivalence (2) follows from Lemma 3.11.9 (ii). 

%%

\section{Proof of Theorem 4.8.3}

Here are some details about the last sentence of the proof of Theorem 4.8.3:

Firstly we rewrite (2.9.4) as follows. Abbreviating transport by $t$ we have 
\begin{equation}\label{2.9.4}
t^{A\to B}(p,f,a_2)=t^B(p,f(t^A(p^{-1},x)))
\end{equation}
for $p:x_1=_Xx_2,\ f:A(x_1)\to B(x_1),\ a_2:A(x_2)$. 

Now let us move to last sentence of the proof of Theorem 4.8.3. Put
$$
F:\equiv\sum_{b:B}\oo{fib}_f(b),
$$ 
and let $p_1:F\to B$ be the first projection. We must show 
\begin{equation}\label{2}
(F,p_1)=_{\sum_{X:\U}(X\to B)}(A,f).
\end{equation} 
We have $e:F\simeq A$. Put $q:\equiv\oo{ua}(e)$. We have $q:F=A$. We claim 
\begin{equation}\label{3}
q_*(p_1)=f.
\end{equation} 
By Theorem 2.7.2, \eqref{3} will imply \eqref{2}. For all $X:\U$ set $I(X)\equiv X,\ C(X)\equiv B$. Then \eqref{3} becomes, in the notation of (2.9.4), 
$$
t^{I\to C}(q,p_1)=f.
$$ 
Let $a:A$. We must show 
$$
t^{I\to C}(q,p_1,a)=f(a).
$$ 
We get 
$$
t^{I\to C}(q,p_1,a)\overset{\text{(a)}}=t^C(q,p_1(t^I(q^{-1},a)))\overset{\text{(b)}}=p_1(e^{-1}(a))\overset{\text{(c)}}=f(a),
$$ 
where (a) follows from \eqref{2.9.4}, (b) follows from the computation rule for univalent stated right after Remark 2.10.4, and (c) follows from the definition of $e^{-1}$.

%%

\section{Proof of Theorem 4.8.4}

The proof of Theorem 4.8.4 uses the following lemma: For $A:\U$ we have 
$$
A\simeq\sum_{X:\U}(X=A)\times X.
$$
More precisely, we shall define maps 
$$
A\rightleftarrows\sum_{X:\U}(X=A)\times X
$$ 
and check that they are mutual quasi-inverses. To $a:A$ we attach $(A,\oo{refl}_A,a)$, and to $(B,p,b)$ we attach $t^{X\mapsto X}(p,b)$. Here and in the sequel we write $t$ for "transport". We have 
$$
a\mapsto(A,\oo{refl}_A,a)\mapsto t^{X\mapsto X}(\oo{refl}_A,a)\equiv a.
$$ 
We have 
$$
(B,p,b)\mapsto t^{X\mapsto X}(p,b)\mapsto(A,\oo{refl}_A,t^{X\mapsto X}(p,b)),
$$ 
and we must show 
$$
(B,p,b)=(A,\oo{refl}_A,t^{X\mapsto X}(p,b)).
$$ 
By Theorem 2.7.2 it suffices to verify 
$$
t^{X\mapsto (X=A)\times X}(p,(p,b))=(\oo{refl}_A,t^{X\mapsto X}(p,b)).
$$ 
By Theorem 2.6.4 it suffices to prove 
$$
t^{X\mapsto (X=A)}(p,p)=\oo{refl}_A.
$$ 
But this follows from Lemma 2.11.2.

%%

\section{Display (6.2.2)}

In the setting of Lemma 2.3.4, we have 
$$
\msf{adp}_f(p):f(x)=^P_pf(y).
$$

%%

\section{Proof of Corollary 6.4.3}

Recall the following facts: 

A type $A$ is a set if and only if $x=y$ is a mere proposition for all $x,y:A$ (observation stated just before Lemma 3.3.4). 

A mere proposition is a set (Lemma 3.3.4).

A type is contractible if and only if it is an inhabited mere proposition (Lemma 3.11.3). 

The proof of Corollary 6.4.3 uses the following fact:

Let $A$ and $B$ be types, let $f,g:A\to B$, let $u:\mathsf{isequiv}(f),\ u:\mathsf{isequiv}(g)$, so that $(f,u),(g,v):A\simeq B$. Put $C:\equiv\mathsf{isequiv}(g), D(p):\equiv(p_*(u)=_Cv)$. By Theorem 2.7.2 we have 
$$
((f,u)=(g,v))\simeq\sum_{p:f=g}D(p).
$$ 
We claim that the first projection 
$$
p_1\sum_{p:f=g}D(p)\to(f=g)
$$ 
is an equivalence. This will imply 
$$
((f,u)=(g,v))\simeq(f=g).
$$ 
By Theorem 4.2.13 and the above reminders, $D(p)$ is contractible for all $p$, and Lemma 3.11.9 (i) implies that $p_1$ is indeed an equivalence.

%%

\section{Proof of Lemma 6.5.1}

The first display follows from Theorem 2.11.3.

%%

\section{Appendix A}

It seems to me that the expression $\lambda x.b$ is defined in two conflicting ways, firstly at the beginning of A.1, and secondly in the last sentence of A.2.4. 

\end{document}
